% ===============================================================
% =							Related Work 						=
% ===============================================================
\chapter{Trabalho relacionado}
\label{cha:related_work}
\todo[inline]{Rever e completar}
Neste capítulo serão apresentados conceitos e técnicas relacionadas com o tema proposto por esta tese. Foram lidos alguns trabalhos de investigação e aplicações em produção relacionados com estes conceitos e técnicas para fundamentar o trabalho que se pretende realizar com este tema. Este capítulo será dividido em duas grandes secções. Na primeira secção está repartida por 3 subsecções onde serão discutidas aplicações e técnicas utilizadas na altura de tirar a fotografia e na pós-produção da mesma. Na segunda secção \todo{quantas subsecções?}serão discutidos técnicas que permitem a avaliação e classificação de fotografias com base em regras de composição.

% # SECTION: Captura e Tratamento de Fotografia #
\section{Captura e Tratamento de Fotografia}
\label{sec:foto_capture_handle}

% # SUBSECTION: Captura de Fotografia #
\subsection{Captura de Fotografia}
\label{sub:foto_capture}

O mercado Android já apresenta uma vasta panóplia de aplicações fotográficas para além das existentes no sistema operativo. 
\todo[inline]{Instagram, Photoshop Express, Photosynth/Photo Sphere, SnapSeed, 6tag, Camera FV5}

% # SUBSECTION: Técnicas de Fotografia #
\subsection{Técnicas de Fotografia}
\label{sub:foto_tecniques}

\todo[inline]{Longe exposure, Panoramic, HDR, Controlar a profundidade do campo (ter tudo em foco ou apenas uma pequena parte em foco.. pode não ser possível), Panning, Tilt-shift, Black and white, Infrared? (não há sensor), Cinemagraph, control de focus, aperture, shutter speed, white balance, metering, iso speed, auto focus}
	
% # SUBSECTION: Pós-Produção de Fotografia #
\subsection{Pós-Produção de Fotografia}
\label{sub:foto_post_prod}

\todo[inline]{Edit options: Instagram, Photoshop Express, Photosynth/Photo Sphere, SnapSeed, 6tag, Camera FV5}

%------------------------------------------------------------------------------------------------------------------
%------------------------------------------------------------------------------------------------------------------
%------------------------------------------------------------------------------------------------------------------

% # SECTION: Avaliação de Fotografia #
\section{Avaliação de Fotografia}
\label{sec:foto_eval}

% # SUBSECTION: Regras de Composição #
\subsection{Regras de Composição}
\label{sub:foto_rules}
