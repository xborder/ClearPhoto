% ===============================================================
% =						Fundamental Concepts					=
% ===============================================================
% # SECTION: Fundamental Concepts #
\chapter{Fundamental Concepts}
In the world of photography, there are many technical concepts and properties that are fundamental in the art of photography. It is the photographer's job to use those concepts and properties in order to explore her creativity and capture the moment with the best possible result.
For a better understanding of this proposal, there are some of those concepts and properties enumerated in the following section, with a brief description.

\section{Light}

Probably the most fundamental element in photography, capturing the light reflected by the objects is the core in photography. The various colors of the light spectrum are reflected and recorded by the cameras sensor, defining an image in a raw format with all the chrominance information

\section{Exposure}

Exposure is the amount of light that reaches the camera’s sensor and is controlled by choosing the shutter-speed, aperture of the lens and ISO value, although ISO doesn’t necessarily affect the amount of light that goes through.

\section{Shutter}

Shutter is an electronic and mechanic component that allows the light to pass for determined period of time, and reach the light-sensitive electronic sensor to capture a permanent image of the scene. The velocity the shutter takes to perform and action is called shutter-speed and this can vary from seconds to microseconds, depending on the technique the photographer intends to use to capture the scenery. E.g., lower shutter-speeds allow to create long exposure images, while faster shutter-speeds tend to avoid shaken or blurred images, allowing perfectly sharp images of objects or people in movement.

\section{Aperture}

Aperture is the hole that controls the cone angle of rays of light and the quantity of light that reaches the sensor. The narrower the aperture, the sharper the photo becomes due to highly collimated rays of light (i.e. rays of light are parallel, and therefore will spread minimally as it propagates).
If the aperture is wide, the result will be an image sharpen around what the lens is focusing on and blurred otherwise.
The aperture values are expressed in integers designed by f/stops, where f stands for focal length. The lower the aperture value, the higher focal length value is.
The aperture does not affect the shutter-speed directly, but it affects the quantity of light that reaches the camera’s sensor, which makes it possible to use lenses with larger aperture values in quick photographs.

\section{Depth of field}

Depth of field is the distance between the nearest and farthest objects in a scene that appear acceptably sharp in an image. This effect can vary with the lens aperture and the size of the camera sensor. In both cases, larger lens apertures and  higher the sensors sensitivity, the lower the depth of field.
The depth of field can be used in a creative manner, leaving to the  photographer's criteria, the amount of sharpness she wants from the nearest object, to the farthest object (Figure X).

\section{ISO}

This is the measure that defines the camera’s sensor sensitivity to the light. Digital cameras tend to behave better in low light conditions with higher ISO values. This means, that for higher ISO values, the camera’s sensor becomes more sensible to light rays.
In digital cameras and mobile devices, the sensitivity can be adjusted if necessary. However, increasing the camera’s sensitivity to the light recklessly might ruin a photograph due to the fact that it will introduce some digital noise in the image, as shown in Figure X. To reduce this negative effect in the image, the use of high ISO values can be compensated with fast shutter speeds and low aperture values.

\section{White-balance}

It is a known fact that the human eye is more sensible to light variations than color variations, therefore, when we see an object reflecting light, our brain instantly interprets the color. This means that in areas of different brightness, our eyes adapt and interpret the same color, although, to the camera they’re not equal.
Since camera’s are not capable of simulating our brain, that is why white-balance is used in professional photography, in order to match the captured ambience light to what our brain would read. Figure X illustrates various examples of images with different tonalities that can be corrected adjusting the white-balance.