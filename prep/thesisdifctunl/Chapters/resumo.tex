\abstractPT 

A massificação de dispositivos móveis, deu a conhecer ao público em geral novas áreas que estavam confinadas a dispositivos especializados. De uma forma geral, o \emph{smartphone} veio dar a todos os seus utilizadores a capacidade de realizar múltiplas tarefas, e entre elas, fotografar com recurso a câmaras integradas.

% What's the problem?

Embora estes dispositivos venham com câmaras cada vez melhores, o seu potencial não é totalmente aproveitado pelas APIs simplificadas dos fabricantes e aplicações simplistas que o utilizador se depara quando fotografa.

% Why is it interesting?

Aproveitando este ambiente propício ao crescimento do computador de bolso, encontra-mo-nos no melhor cenário para desenvolver aplicações que ajudem o utilizador a obter um bom resultado.

% What's the solution?

Como tentativa de fornecer algo mais aplicado à tarefa em questão, surge esta dissertação, que apresenta como contribuição uma aplicação para dispositivos móveis capaz de fornecer informações em real-time sobre o enquadramento do cenário a ser capturado, anotação e aplicação de efeitos sobre a imagem capturada.


% What follows from the solution?

Assim, a solução proposta tem como objetivo o desenvolvimento de uma interface que suporte captura e edição de imagem usando um dispositivo móvel. O utilizador terá a possibilidade de personalizar alguns parâmetros fotográficos disponíveis pelo sistema de captura de imagem e receber sugestões sobre a composição de imagem, que será avaliado segundo regras de composição de fotografia. Por semelhança ao fluxo de trabalho de fotógrafos profissionais, após a captura, o utilizador poderá também adicionar notas e aplicar um conjunto de correções e filtros sobre a imagem capturada.

% Palavras-chave do resumo em Português
\begin{keywords}
Fotografia, câmaras, estética, captura de imagem, edição de imagem, qualidade de imagem, dispositivos móveis.
\end{keywords}
% to add an extra black line
