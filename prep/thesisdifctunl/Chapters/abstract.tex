\abstractEN

The widespread of mobile devices, has made known to the general public new areas that were hitherto confined to specialized devices. In general, the smartphone came to give all users the ability to realize multiple tasks, and among them, take photographs using the integrated cameras.

% What's the problem?

Although these devices are continuously receiving improved cameras, their manufacturers do not take advantage of its full potential, by offering simplified APIs and simplistic applications that users face when shooting.

% Why is it interesting?

Taking advantage of this growth-conducive environment for mobile devices, we find ourselves in the best scenario to develop applications that help the user obtaining a good result when shooting.

% What's the solution?

In an attempt to provide something more applied to the task, this dissertation presents as a contribution, an application for mobile devices that provide information in real-time on the composition of the scene being captured, and allows a user to annotate and apply effects on the captured image.

% What follows from the solution?
Thus, the proposed solution aims to develop an interface that supports capture and image editing using a mobile device. The user will be able to customize some photographic parameters available by the imaging system and receive suggestions on the composition of the scene, which will be evaluated according to rules of photography composition. Similarly to the work-flow of professional photographers, after capture a user can also add notes, and apply a set of patches and filters on the captured image.

% Palavras-chave do resumo em Inglês
\begin{keywords}
Photography, cameras, aesthetic, image capture, image editing, image quality, multimedia content, mobile.
\end{keywords}
% to add an extra black line
