\abstractPT 
A massificação de dispositivos móveis, deu a conhecer ao público em geral novas áreas que estavam confinadas a dispositivos especializados. De uma forma geral, o \emph{smartphone} veio dar a todos os seus utilizadores a capacidade de realizar múltiplas tarefas, e entre elas, fotografar com recurso a câmaras integradas.

% What's the problem?

Embora estes dispositivos venham com câmaras cada vez melhores, o seu potencial não é totalmente aproveitado pelas APIs simplificadas dos fabricantes e aplicações simplistas que o utilizador se depara quando fotografa. Aproveitando este ambiente propício ao crescimento do computador de bolso, encontra-mo-nos no melhor cenário para desenvolver aplicações que ajudem o utilizador a obter um bom resultado.

% What's the solution?

Com o objectivo de fornecer um conjunto de técnicas e ferramentas aplicadas à tarefa em questão, surge esta dissertação, que apresenta como contribuição uma aplicação para dispositivos móveis capaz de fornecer informações em real-time sobre o enquadramento do cenário a ser capturado, anotação e aplicação de efeitos sobre a imagem capturada.


% What follows from the solution?

Assim, a solução proposta visa o desenvolvimento de um aplicativo que dá suporte a um utilizador durante a captura de uma cena com ela dispositivo móvel. O utilizador será capaz de receber várias sugestões sobre a composição da cena, que será baseado em regras de fotografia ou ferramentas úteis para os fotógrafos. Esta ferramenta incluem detecção horizonte e visualização gráfica de uma paleta de cores apresentada no visor. A avaliação será feita sobre a implementação num dispositivo móvel e como os utilizadores avaliam a sua utilidade.

% Palavras-chave do resumo em Português
\begin{keywords}
Fotografia, câmaras, estética, captura de imagem, qualidade de imagem, dispositivos móveis.
\end{keywords}
% to add an extra black line