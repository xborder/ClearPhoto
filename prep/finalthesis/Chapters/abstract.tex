\abstractEN
\todo[inline]{modificado, para revisão}

The widespread of mobile devices, has made known to the general public new areas that were hitherto confined to specialized devices. In general, the smartphone came to give all users the ability to realize multiple tasks, and among them, take photographs using the integrated cameras.

% What's the problem?

Although these devices are continuously receiving improved cameras, their manufacturers do not take advantage of its full potential, by offering simplified APIs and simplistic applications for shooting.

% Why is it interesting?

Taking advantage of this growth-conducive environment for mobile devices, we find ourselves in the best scenario to develop applications that help the user obtaining a good result when shooting.

% What's the solution?

In an attempt to provide something more applied to the task, this dissertation presents as a contribution, an application for mobile devices that provides information in real-time on the composition of the scene before capturing an image.

% What follows from the solution?
Thus, the proposed solution aims to develop an application that gives support to an user while capturing a scene with her mobile device. The user will be able to receive multiple suggestions on the composition of the scene, which will be based on rules of photography or useful tools for photographers. This tools include horizon detection and graphical visualization of the color palette presented on the scenario being photographed.

% Palavras-chave do resumo em Inglês
\begin{keywords}
Photography, cameras, aesthetic, image capture, image quality, mobile devices.
\end{keywords}
% to add an extra black line
