\abstractEN
The widespread use of mobile devices, has made known to the general public new areas that were hitherto confined to specialized devices. In general, the smartphone came to give all users the ability to execute multiple tasks, and among them, take photographs using the integrated cameras.

% What's the problem?

Although these devices are continuously receiving improved cameras, their manufacturers do not take advantage of its full potential, since the operating systems normally offer simple APIs and applications for shooting. Therefore, taking advantage of this environment for mobile devices, we find ourselves in the best scenario to develop applications that help the user obtaining a good result when shooting.

% What's the solution?

In an attempt to provide a set of techniques and tools more applied to the task, this dissertation presents as a contribution, a set of tools for mobile devices that provides information in real-time on the composition of the scene before capturing an image.

% What follows from the solution?
Thus, the proposed solution gives support to an user while capturing a scene with her mobile device. The user will be able to receive multiple suggestions on the composition of the scene, which will be based on rules of photography or other useful tools for photographers. The tools include horizon detection and graphical visualization of the color palette presented on the scenario being photographed. These tools were evaluated regarding the mobile device implementation and how users assess their usefulness.

% Palavras-chave do resumo em Inglês
\begin{keywords}
Photography, cameras, aesthetic, image capture, image quality, mobile devices.
\end{keywords}
% to add an extra black line
