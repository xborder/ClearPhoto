\chapter{Conclusion and Future Work}
\label{cha:conclusion_future}

The following chapter will present a brief analysis concerning the work accomplished with this dissertation, as well as what we pretend to improve and implement in future applications.

\section{Conclusion}
\label{sec:conclusion}

This dissertation introduces a set a features based on rules used by professional photographers, applied in a real-time scenario when photographing. This was developed as a proof-of-concept for a smartphone.

After a thorough research, we could conclude that colour and composition were important properties to attain a better aesthetic result in photography. This solution provides means for gathering information about the scenario being photographed using these properties. Using these tools, the user is now able to understand what are the most used colours, if the scenario is using monochromatic or complementary tones, and the pureness of the colours being used. With this, the user can try and balance the colours.

The same goes for composition, as this thesis offers tools to detect important elements such as the horizon line, the subject of a scenario and detect its simplicity. As a helping tool to obtain an aesthetic result, the user can then take advantage of the suggestions and reorganize the composition of the scenario. 

In our testing stage, we compared our modified versions of algorithms with the original ones, we tested the performance of each feature and realized user questionnaires to assess its usefulness and accuracy. From the tests realized, we could conclude that these algorithms can be useful but there is still many room for improvement. A more user oriented interface, faster algorithms and better devices, would be needed.

It is important to refer that the final decision about the composition and colours is always of the final user. Although these features try to give a suggestion based on a set of rules that are considered to, it is important to know when to ignore the suggestions and break the rules.

A limitation of this tool is the device in which it runs. Some of the features implemented are computationally heavy which makes it not perfect to be used in a real-time scenario and loses accuracy. This was confirmed by the performance tests that we realized.

There are still many improvements and functionalities that could be implemented, which is the subject of the following section. Aside from the developed work during the course of this dissertation, writing a paper for submitting in a human-machine interaction or multimedia international conference is also planned.

\section{Future Work}
\label{sec:future}

In the future, some of the features in our solution must be improved since running these algorithms in real-time is a heavy task for a mobile device. Including an external processing unit to the architecture, should be an option to take into consideration, as it could be a way to reduce the processing times.

Since the purpose of this thesis is to give some type of information about the aesthetics of a photo, in the future, we will implement a classification system to rate the photo after it was taken as it would encourage the user to try and understand the flaws in the scenario and correct them.

In the end, this set of tools would be fully integrated in a photographic application along with the a set of utilities for photo manipulation to be used after the capture. This utilities would include tools such as rotate, crop, apply a set of available filters, and auto white-balance that could be applied to specific parts of the image. If a more capable manipulation software was needed, after the capture, the user could annotate directly on the photo which areas should be corrected and what correction to apply. In this case, a subset of handwritten words that related to image corrections could be detected and applied directly. A more user oriented interface with hints explaining each feature would also be necessary to give the final user an idea of what expect from it.

Finally, we believe that our solution contributed to this field and could be further improved and so the writing of a paper, or papers, would follow the delivery of this document.