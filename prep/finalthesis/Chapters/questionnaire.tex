\documentclass[a4paper,10pt,BCOR10mm,oneside,headsepline]{scrartcl}
\usepackage[ngerman]{babel}
\usepackage[utf8]{inputenc}
\usepackage{wasysym}% provides \ocircle and \Box
\usepackage{enumitem}% easy control of topsep and leftmargin for lists
\usepackage{color}% used for background color
\usepackage{forloop}% used for \Qrating and \Qlines
\usepackage{ifthen}% used for \Qitem and \QItem
\usepackage{typearea}
\areaset{17cm}{26cm}
\setlength{\topmargin}{-1cm}
\usepackage{scrpage2}
\pagestyle{scrheadings}
\ihead{ClearPhoto - Questionnaire}
\ohead{\pagemark}
\chead{}
\cfoot{}

%%%%%%%%%%%%%%%%%%%%%%%%%%%%%%%%%%%%%%%%%%%%%%%%%%%%%%%%%%%%
%% Beginning of questionnaire command definitions %%
%%%%%%%%%%%%%%%%%%%%%%%%%%%%%%%%%%%%%%%%%%%%%%%%%%%%%%%%%%%%
%%
%% 2010, 2012 by Sven Hartenstein
%% mail@svenhartenstein.de
%% http://www.svenhartenstein.de
%%
%% Please be warned that this is NOT a full-featured framework for
%% creating (all sorts of) questionnaires. Rather, it is a small
%% collection of LaTeX commands that I found useful when creating a
%% questionnaire. Feel free to copy and adjust any parts you like.
%% Most probably, you will want to change the commands, so that they
%% fit your taste.
%%
%% Also note that I am not a LaTeX expert! Things can very likely be
%% done much more elegant than I was able to. If you have suggestions
%% about what can be improved please send me an email. I intend to
%% add good tipps to my website and to name contributers of course.
%%
%% 10/2012: Thanks to karathan for the suggestion to put \noindent
%% before \rule!

%% \Qq = Questionaire question. Oh, this is just too simple. It helps
%% making it easy to globally change the appearance of questions.
\newcommand{\Qq}[1]{\textbf{#1}}

%% \QO = Circle or box to be ticked. Used both by direct call and by
%% \Qrating and \Qlist.
\newcommand{\QO}{$\Box$}% or: $\ocircle$

%% \Qrating = Automatically create a rating scale with NUM steps, like
%% this: 0--0--0--0--0.
\newcounter{qr}
\newcommand{\Qrating}[1]{\QO\forloop{qr}{1}{\value{qr} < #1}{---\QO}}

%% \Qline = Again, this is very simple. It helps setting the line
%% thickness globally. Used both by direct call and by \Qlines.
\newcommand{\Qline}[1]{\noindent\rule{#1}{0.6pt}}

%% \Qlines = Insert NUM lines with width=\linewith. You can change the
%% \vskip value to adjust the spacing.
\newcounter{ql}
\newcommand{\Qlines}[1]{\forloop{ql}{0}{\value{ql}<#1}{\vskip0em\Qline{\linewidth}}}

%% \Qlist = This is an environment very similar to itemize but with
%% \QO in front of each list item. Useful for classical multiple
%% choice. Change leftmargin and topsep accourding to your taste.
\newenvironment{Qlist}{%
\renewcommand{\labelitemi}{\QO}
\begin{itemize}[leftmargin=1.5em,topsep=-.5em]
}{%
\end{itemize}
}

%% \Qtab = A "tabulator simulation". The first argument is the
%% distance from the left margin. The second argument is content which
%% is indented within the current row.
\newlength{\qt}
\newcommand{\Qtab}[2]{
\setlength{\qt}{\linewidth}
\addtolength{\qt}{-#1}
\hfill\parbox[t]{\qt}{\raggedright #2}
}

%% \Qitem = Item with automatic numbering. The first optional argument
%% can be used to create sub-items like 2a, 2b, 2c, ... The item
%% number is increased if the first argument is omitted or equals 'a'.
%% You will have to adjust this if you prefer a different numbering
%% scheme. Adjust topsep and leftmargin as needed.
\newcounter{itemnummer}
\newcommand{\Qitem}[2][]{% #1 optional, #2 notwendig
\ifthenelse{\equal{#1}{}}{\stepcounter{itemnummer}}{}
\ifthenelse{\equal{#1}{a}}{\stepcounter{itemnummer}}{}
\begin{enumerate}[topsep=2pt,leftmargin=2.8em]
\item[\textbf{\arabic{itemnummer}#1.}] #2
\end{enumerate}
}

%% \QItem = Like \Qitem but with alternating background color. This
%% might be error prone as I hard-coded some lengths (-5.25pt and
%% -3pt)! I do not yet understand why I need them.
\definecolor{bgodd}{rgb}{0.8,0.8,0.8}
\definecolor{bgeven}{rgb}{0.9,0.9,0.9}
\newcounter{itemoddeven}
\newlength{\gb}
\newcommand{\QItem}[2][]{% #1 optional, #2 notwendig
\setlength{\gb}{\linewidth}
\addtolength{\gb}{-5.25pt}
\ifthenelse{\equal{\value{itemoddeven}}{0}}{%
\noindent\colorbox{bgeven}{\hskip-3pt\begin{minipage}{\gb}\Qitem[#1]{#2}\end{minipage}}%
\stepcounter{itemoddeven}%
}{%
\noindent\colorbox{bgodd}{\hskip-3pt\begin{minipage}{\gb}\Qitem[#1]{#2}\end{minipage}}%
\setcounter{itemoddeven}{0}%
}
}

%%%%%%%%%%%%%%%%%%%%%%%%%%%%%%%%%%%%%%%%%%%%%%%%%%%%%%%%%%%%
%% End of questionnaire command definitions %%
%%%%%%%%%%%%%%%%%%%%%%%%%%%%%%%%%%%%%%%%%%%%%%%%%%%%%%%%%%%%

\begin{document}

\section*{User Data}

\Qitem{ \Qq{Gener?} M \QO{} F \QO{}}
\Qitem{ \Qq{Age?} \Qline{1.5cm}}

\section*{User's Past Experience}
\Qitem{ \Qq{Do you take photographs?} \hskip0.4cm Yes \QO{} \hskip0.5cm No \QO{} }

\Qitem{ \Qq{Do you have any knowledge in photography?} \hskip0.4cm Yes \QO{} \hskip0.5cm No \QO{} }

\Qitem{ \Qq{If yes, how do you classify your knowledge in photography?} \hskip0.4cm Professional \QO{} \hskip0.5cm Amateur \QO{} }


\Qitem{ \Qq{If yes, what kind of device do you normally use to take photographs?}
\begin{Qlist}
\item Still cameras \Qtab{3cm}{Never \Qrating{5} Regularly}
\item Instant cameras  \Qtab{3cm}{Never \Qrating{5} Regularly}
\item DSLR cameras  \Qtab{3cm}{Never \Qrating{5} Regularly}
\item Camera phone  \Qtab{3cm}{Never \Qrating{5} Regularly}
\item Other: \Qline{4cm} \Qtab{3cm}{Never \Qrating{5} Regularly}
\end{Qlist}
}

\section*{About this questionnaire}

\minisec{RGB Histograms}
\vskip.5em

\Qitem{ \Qq{The purpose of the dynamic bar is clear }}
\Qtab{3cm}{Strongly disagree \Qrating{5} Strongly agree}

\Qitem{ \Qq{The purpose of the growing line is clear }}
\Qtab{3cm}{Strongly disagree \Qrating{5} Strongly agree}

\Qitem{ \Qq{Gives a good idea of the range of colours being used }}
\Qtab{3cm}{Strongly disagree \Qrating{5} Strongly agree}

\Qitem{ \Qq{This tool is useful in a real scenario}}
\Qtab{3cm}{Strongly disagree \Qrating{5} Strongly agree}

\minisec{Hue Colour Histogram}
\vskip.5em

\Qitem{ \Qq{The number of colours in the spectrum is adequate}}
\Qtab{3cm}{Strongly disagree \Qrating{5} Strongly agree}

\Qitem{ \Qq{Can easily see the most used colour and which complementary colour should be used to balance the image}}
\Qtab{3cm}{Strongly disagree \Qrating{5} Strongly agree}

\Qitem{ \Qq{Gives a good idea of the range of colours being used}}
\Qtab{3cm}{Strongly disagree \Qrating{5} Strongly agree}

\Qitem{ \Qq{This tool is useful in a real scenario}}
\Qtab{3cm}{Strongly disagree \Qrating{5} Strongly agree}


\minisec{Saturation Detection}
\vskip.5em

\Qitem{ \Qq{The visual cue for this feature encourages the usage of a monochromatic filter}}
\Qtab{3cm}{Strongly disagree \Qrating{5} Strongly agree}

\Qitem{ \Qq{This tool is useful in a real scenario}}
\Qtab{3cm}{Strongly disagree \Qrating{5} Strongly agree}


\minisec{Colour Template Detector}
\vskip.5em

\Qitem{ \Qq{The colour scale is adequate}}
\Qtab{3cm}{Strongly disagree \Qrating{5} Strongly agree}

\Qitem{ \Qq{This feature foes along well with the hue scoring feature}}
\Qtab{3cm}{Strongly disagree \Qrating{5} Strongly agree}

\Qitem{ \Qq{It is easy to understand the difference between scenarios with a monochromatic or complementary colour schemes}}
\Qtab{3cm}{Strongly disagree \Qrating{5} Strongly agree}

\Qitem{ \Qq{This tool is useful in a real scenario}}
\Qtab{3cm}{Strongly disagree \Qrating{5} Strongly agree}


\minisec{Hue Count Score}
\vskip.5em

\Qitem{ \Qq{The scoring scale is adequate}}
\Qtab{3cm}{Strongly disagree \Qrating{5} Strongly agree}

\Qitem{ \Qq{This feature goes along well with the colour template detection}}
\Qtab{3cm}{Strongly disagree \Qrating{5} Strongly agree}

\Qitem{ \Qq{The scoring can reflect the simplicity of a scenario}}
\Qtab{3cm}{Strongly disagree \Qrating{5} Strongly agree}

\Qitem{ \Qq{This tool is useful in a real scenario}}
\Qtab{3cm}{Strongly disagree \Qrating{5} Strongly agree}

\minisec{Face Detection}
\vskip.5em

\Qitem{ \Qq{This feature is useful when used with rule of thirds or golden rule of thirds}}
\Qtab{3cm}{Strongly disagree \Qrating{5} Strongly agree}

\Qitem{ \Qq{The connection between three faces is useful to explore the rule of odds}}
\Qtab{3cm}{Strongly disagree \Qrating{5} Strongly agree}

\Qitem{ \Qq{The connections between three faces is useful to explore the triangular composition}}
\Qtab{3cm}{Strongly disagree \Qrating{5} Strongly agree}

\Qitem{ \Qq{The suggestive placements are useful}}
\Qtab{3cm}{Strongly disagree \Qrating{5} Strongly agree}

\Qitem{ \Qq{When used with a grid, it is too much visual information}}
\Qtab{3cm}{Strongly disagree \Qrating{5} Strongly agree}


\Qitem{ \Qq{This tool is useful in a real scenario}}
\Qtab{3cm}{Strongly disagree \Qrating{5} Strongly agree}


\minisec{Horizon Detection}
\vskip.5em

\Qitem{ \Qq{The resulting line gives the expected horizon line}}
\Qtab{3cm}{Strongly disagree \Qrating{5} Strongly agree}

\Qitem{ \Qq{This feature performs well in real-time}}
\Qtab{3cm}{Strongly disagree \Qrating{5} Strongly agree}

\Qitem{ \Qq{This tool is useful in a real scenario}}
\Qtab{3cm}{Strongly disagree \Qrating{5} Strongly agree}

\minisec{Main Lines Detection}
\vskip.5em

\Qitem{ \Qq{The resulting lines correspond to the leading lines in the scenario}}
\Qtab{3cm}{Strongly disagree \Qrating{5} Strongly agree}

\Qitem{ \Qq{The visual cue facilitates the usage of leading lines in a composition}}
\Qtab{3cm}{Strongly disagree \Qrating{5} Strongly agree}

\Qitem{ \Qq{This tool is useful in a real scenario}}
\Qtab{3cm}{Strongly disagree \Qrating{5} Strongly agree}

\minisec{Generic Object Segmentation}
\vskip.5em

\Qitem{ \Qq{The object segmentation is accurate}}
\Qtab{3cm}{Strongly disagree \Qrating{5} Strongly agree}

\Qitem{ \Qq{This tool is useful in a real scenario}}
\Qtab{3cm}{Strongly disagree \Qrating{5} Strongly agree}

\minisec{Image Simplicity Detection}
\vskip.5em

\Qitem{ \Qq{Bars are a good visual indicator for this feature}}
\Qtab{3cm}{Strongly disagree \Qrating{5} Strongly agree}

\Qitem{ \Qq{Numbers are a good visual indicator for this feature}}
\Qtab{3cm}{Strongly disagree \Qrating{5} Strongly agree}

\Qitem{ \Qq{The first method displays a good result for simplicity of the scenario}}
\Qtab{3cm}{Strongly disagree \Qrating{5} Strongly agree}

\Qitem{ \Qq{The second method displays a good result for simplicity of the scenario}}
\Qtab{3cm}{Strongly disagree \Qrating{5} Strongly agree}

\Qitem{ \Qq{The third method displays a good result for simplicity of the scenario}}
\Qtab{3cm}{Strongly disagree \Qrating{5} Strongly agree}

\Qitem{ \Qq{This tool is useful in a real scenario}}
\Qtab{3cm}{Strongly disagree \Qrating{5} Strongly agree}

\minisec{Image Balance Detection}
\vskip.5em

\Qitem{ \Qq{The symmetry axis is accurate}}
\Qtab{3cm}{Strongly disagree \Qrating{5} Strongly agree}

\Qitem{ \Qq{The visual cue displays a correct result from what was expected}}
\Qtab{3cm}{Strongly disagree \Qrating{5} Strongly agree}

\Qitem{ \Qq{The symmetry axis is accurate}}
\Qtab{3cm}{Strongly disagree \Qrating{5} Strongly agree}

\Qitem{ \Qq{This tool is useful in a real scenario}}
\Qtab{3cm}{Strongly disagree \Qrating{5} Strongly agree}

\section*{Suggestions and Comments}

\Qitem{ \Qq{Any comments and suggestions are appreciated.} \Qlines{6} }
\end{document}
